% Appendix A

\chapter{Computation of DCT}
\label{AppendixA}
\lhead{Appendix A. \emph{Computation of DCT}}

The Discrete Cosine Transform (DCT) is a widely used transformation technique in signal processing and data compression. It converts a sequence of input values into a set of frequency coefficients, which represent the signal's energy in different frequency bands. 
Following is the step-by-step computation of the DCT for a 3x3 matrix.

Step 1: Define the input matrix
\[
\mathbf{A} = \begin{bmatrix}
	a_{1,1} & a_{1,2} & a_{1,3} \\
	a_{2,1} & a_{2,2} & a_{2,3} \\
	a_{3,1} & a_{3,2} & a_{3,3} \\
\end{bmatrix}
\]

Step 2: Subtract the mean from the matrix
\[
\bar{\mathbf{A}} = \begin{bmatrix}
	a_{1,1} - \mu & a_{1,2} - \mu & a_{1,3} - \mu \\
	a_{2,1} - \mu & a_{2,2} - \mu & a_{2,3} - \mu \\
	a_{3,1} - \mu & a_{3,2} - \mu & a_{3,3} - \mu \\
\end{bmatrix}
\]
where \(\mu\) is the mean value of the elements in matrix \(\mathbf{A}\).

Step 3: Compute the DCT coefficients
\[
\mathbf{B} = \begin{bmatrix}
	b_{1,1} & b_{1,2} & b_{1,3} \\
	b_{2,1} & b_{2,2} & b_{2,3} \\
	b_{3,1} & b_{3,2} & b_{3,3} \\
\end{bmatrix}
\]
where
\[
b_{i,j} = \sum_{x=1}^{3} \sum_{y=1}^{3} \bar{a}_{x,y} \cdot \cos\left(\frac{(2x-1)(i-1)\pi}{6}\right) \cdot \cos\left(\frac{(2y-1)(j-1)\pi}{6}\right)
\]
for \(i,j = 1,2,3\).

Step 4: Normalize the DCT coefficients
\[
\mathbf{C} = \begin{bmatrix}
	c_{1,1} & c_{1,2} & c_{1,3} \\
	c_{2,1} & c_{2,2} & c_{2,3} \\
	c_{3,1} & c_{3,2} & c_{3,3} \\
\end{bmatrix}
\]
where
\[
c_{i,j} = \frac{1}{\sqrt{2N}} \cdot b_{i,j}
\]
for \(i,j = 1,2,3\) and \(N\) is the size of the matrix (in this case, \(N = 3\)).

